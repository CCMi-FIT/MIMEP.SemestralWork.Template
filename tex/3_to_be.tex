\chapter{Digitalization Analysis (To-Be)}\label{sec:to-be}

%TODO: remove the instructions

This is the most critical phase for IT entrepreneurs, where real value is created. Now that we understand the domain, we can propose innovative software solutions to support and enhance the process. 

\begin{enumerate}
    \item Create a To-Be BPMN Level 2 model.
        \begin{enumerate}
            \item At least 20 BPMN activities.
            \item You are free to introduce changes — the future process may differ from the current (As-Is) process and may include hypothetical improvements.
        \end{enumerate}
    \item Create forms description for all the BPMN activities.
     \begin{enumerate}
            \item Provide a textual description of each form, including:
            \begin{enumerate
            	\item Input values expected from the user or system.
            	\item Outputs or data displayed in the form.
            \end{enumerate}
            \item imple validation rules should be included where applicable. Wireframes are not required.
            \item The forms should collectively cover at least 75\% of the Process Structure Diagram (PSD).
        \end{enumerate}
    \item Summarize the improvements. Highlight the key changes and their expected benefits (e.g., efficiency, automation, user experience).
\end{enumerate}

\section{Analytical To-Be Models}
%The BPMN model figures go here. 

\section{Forms}
%The forms go here. 

\section{Summary}
%The world peace will be achieved with the proposed changes. 
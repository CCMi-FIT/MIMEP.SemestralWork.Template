\chapter{Organization Essence Revealing (As-Is)}

%TODO: remove the instructions

The goals of this chapter are to perform an Organisation Essence Revealing (OER) analysis as described in~\cite{dietz2015teoo,dietz2020enterprise}. We will only do the following steps: 

\begin{enumerate}
	\item Provide the domain description and perform the OER analysis in order to identify transactions.
	\begin{enumerate}
		\item Write the project domain description into this document. The domain should have at least 5 ontological transactions. Each transaction needs to have defined at least 5 act.
		\item Perform the OER analysis and find \oact{ontological acts}, \iact{informational acts} and \dact{documental acts}. 
		\item Identify all transaction kinds and explicitly highlight them in the text, including their identification (preferably numerical) and the specific ontological act involved in the transaction. For example, \oact{[TK1/rq]} indicates that the ontological act is a \textit{request} belonging to \textit{transaction kind no. 1}.
	\end{enumerate}
    \item Create an extended Transaction Result Table (e-TRT). Map the identified ontological acts from Step 2 to their corresponding transactions and specific transaction acts, organizing them into coherent and complete transaction structures. Refer to~\cref{tab:etrt} for an example. For each transaction, also provide a properly formulated name, its resulting product, and the (DEMO) actor roles of both the initiator and executor. If any transaction act lacks a corresponding ontological act identified in the domain description, it may be left empty—however, indicate explicitly that it is missing and verify carefully that this is indeed the case.
    \item Create a Subject-Actor Table to clearly distinguish between domain-specific roles and their corresponding DEMO actor roles within the transactions. See the~\cref{tab:subjectactortable}.
    \item Produce diagrams, all needs to be in English.
    \begin{enumerate}
    	\item Produce the Coordination Structure Diagram (CSD) at the ontological level only, see the~\cref{fig:csdModel}.
    	\item Produce the Process Structure Diagram (PSD), see the~~\cref{fig:psdModel}.
    	\item Produce the Object Fact Diagram (OFD). The result should be detailed enough for someone to derive a database model from it. Ensure it includes at least 5 entities and 15 attributes. See the~~\cref{fig:ofdModel}.
    \end{enumerate}
    \item Summarize your modeling thoughts and revelations. 
\end{enumerate}

\textbf{Correct models are not created on the first iteration, one must go through the steps many times, combine and split transactions to achieve the final result. }. 

\section{OER Step 1: Distinguishing Performa-Informa-Forma}

Legend: 
\begin{itemize}
    \item \oact{Ontological Act [Transaction Kind/Act type]}
    \item \iact{Informational Act}
    \item \dact{Documental Act}
\end{itemize}

The process description of the Volley Case and it's OER analysis was taken from the Enterprise Ontology book~\cite{dietz2020enterprise}

\paragraph{\S 1 Preliminary Rules}

\begin{enumerate}[label={(\arabic*)}]
\item One can \oact{become member of the tennis club Volley[TK1]} by \oact{sending a letter}\oact{[TK1/rq]} to the club by \iact{postal mail}. In the letter one has to mention \iact{one’s surname and first name, birth date, gender, telephone number, and postal mail address (street, house number, zip code, and town)}. Adam, the administrator of Volley, \dact{empties the mailbox} daily and checks whether the information provided is complete. If not, he \dact{makes a telephone call} to the sender in order \iact{to complete the data}. Once a letter is complete, Adam \dact{writes an incoming mail number and the date on the letter, records the letter in the letter book, and puts it in a folder}. 
\item Every Wednesday evening, Adam \dact{takes the folder} to Eve, the secretary of Volley. He also \dact{takes the member register with him}. If Eve \oact{decides that an applicant can become member of Volley[TK1/pm]}, \dact{she stamps ‘new member’ on the letter and writes the date below it. She then hands the letter to Adam in order to add the new member to the member register. This is a book with numbered lines. Each new member is entered on a new line. The line number is the number by which the new member is referenced in the administration}. Next, Eve \iact{calculates the fee} that the new member \oact{has to pay [TK2]} for the remaining part of the calendar year. She asks Adam for the \iact{annual fee}, as \oact{decided at the general assembly [TK out of scope]}, which Adam \dact{has recorded on a sheet of paper}. Then, she asks Adam to \dact{write down the amount in the member register}. 
\item If Eve \oact{does not allow an applicant to become member[TK1/dc]} (e.g. because he or she is too young or because the maximum number of members has been reached), Adam will \oact{send a letter}\oact{[TK2/rq]} in which he \iact{explains why the applicant cannot (yet) become member of Volley}.
\end{enumerate}

\paragraph{\S 2 Some Other Rules}

\begin{enumerate}[label={(\arabic*)}]
\item When all applications are processed, Adam \dact{takes the letters and the member register home} and \iact{prepares an invoice} to all new members for the \oact{payment of the first fee[TK2]}. He \oact{sends these invoices}\oact{[TK2/rq]} \dact{by postal mail}. Payments have to be performed by bank transfers.
\item As soon as \oact{a bank statement is received}\oact{[TK2/da]}, Adam \dact{prints a card} on which \iact{the member number, the starting date, the name, the date of birth, the gender, and the residence} are mentioned. \oact{The card is sent}\oact{[TK1/da]} \dact{to the new member by postal mail}.
\end{enumerate}


\begin{landscape}
\section{OER Step 2: Identifying Transaction Kinds}

\begin{table}[h]
\caption{Extended Transaction Result Table}
\label{tab:etrt}
\begin{tabular}{|l||l|l|}
\hline
Transaction  & Membership Starting (TK1) & Membership Paying (TK2) \\ \hline
Product      & membership is started  & the first fee of membership is paid \\ \hline
Initiator      & Aspirant Member (AR1)   &  Membership Starter (AR2)\\ \hline
Executor       & Membership Starter (AR2) & Membership Payer (AR3)       \\ \hline
Request        & Sending a letter (\S1/1)  & Sends the invoices (\S2/1)   \\ \hline
Promise        &  Application decision  (\S1/2)  &  Not Specified    \\ \hline
Decline        &  Does not allow an applicant to become member (\S1/3)  & Not Specified  \\ \hline
Declare        & The card is sent to the member (\S2/2) & A bank statement is received  (\S2/2) \\ \hline
Reject         &  Not Specified             &  Not Specified   \\ \hline
Accept         & Not Specified  &  Not Specified  \\ \hline
Revoke Request & Not Specified                   & Not Specified        \\ \hline
Revoke Promise & Not Specified                   &  Not Specified       \\ \hline
Revoke Declare & Not Specified                    &  Not Specified      \\ \hline
Revoke Accept  &  Not Specified             &   Not Specified             \\ \hline
\end{tabular}
\end{table}

\section{OER Step 2: Identifying Actor Roles}

\begin{table}[h]
\caption{Subject Actor Table}
\label{tab:subjectactortable}
\begin{tabular}{|l|l|l|l|}
\hline
  & Aspirant Member (AR1)       & Membership Starter (AR2)  & Membership Payer (AR3)  \\ \hline
Administrator   &  & X  &   \\ \hline
Customer & X &  & X \\ \hline
\end{tabular}
\end{table}

\end{landscape}

\section{OER Step 3: Producing the diagrams}

\subsection{Coordination Structure Diagram}

\begin{figure}[h]\centering
	\includegraphics[width=5cm]{pic/VolleyCSD}
	\caption{A CSD Model of Volley~\cite{dietz2020enterprise}}
	\label{fig:csdModel}
\end{figure}

\subsection{Process Structure Diagram}

\begin{figure}[h]\centering
	\includegraphics[width=8cm]{pic/VolleyPSD}
	\caption{A PSD Model of Volley~\cite{dietz2020enterprise}}
	\label{fig:psdModel}
\end{figure}

\subsection{Object Fact Diagram}

\begin{figure}[h]\centering
	\includegraphics[width=12cm]{pic/VolleyOFD.png}
	\caption{An Object Fact Diagram of Volley~\cite{dietz2020enterprise}}
	\label{fig:ofdModel}
\end{figure}


\section{Summary}

Summarize the analysis
